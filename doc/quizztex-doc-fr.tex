% !TeX TXS-program:compile = txs:///arara
% arara: pdflatex: {shell: yes, synctex: no, interaction: batchmode}
% arara: pdflatex: {shell: yes, synctex: no, interaction: batchmode} if found('log', '(undefined references|Please rerun|Rerun to get)')

\documentclass[french,a4paper,11pt]{article}
\usepackage[margin=2cm,includefoot]{geometry}
\def\TPversion{0.1.0}
\def\TPdate{12 juin 2023}
\usepackage[utf8]{inputenc}
\usepackage[T1]{fontenc}
\usepackage{amsmath,amssymb}
\usepackage{quizztex}
\usepackage{awesomebox}
\usepackage{fontawesome5}
\usepackage{footnote}
\makesavenoteenv{tabular}
\usepackage{enumitem}
\usepackage{tabularray}
\usepackage{wrapstuff}
\usepackage{lipsum}
\usepackage{fancyvrb}
\usepackage{fancyhdr}
\fancyhf{}
\renewcommand{\headrulewidth}{0pt}
\lfoot{\sffamily\small [quizztex]}
\cfoot{\sffamily\small - \thepage{} -}
\rfoot{\hyperlink{matoc}{\small\faArrowAltCircleUp[regular]}}

%\usepackage{hvlogos}
\usepackage{hologo}
\providecommand\tikzlogo{Ti\textit{k}Z}
\providecommand\TeXLive{\TeX{}Live\xspace}
\providecommand\PSTricks{\textsf{PSTricks}\xspace}
\let\pstricks\PSTricks
\let\TikZ\tikzlogo
\newcommand\TableauDocumentation{%
	\begin{tblr}{width=\linewidth,colspec={X[c]X[c]X[c]X[c]X[c]X[c]},cells={font=\sffamily}}
		{\LARGE \LaTeX} & & & & &\\
		& {\LARGE \hologo{pdfLaTeX}} & & & & \\
		& & {\LARGE \hologo{LuaLaTeX}} & & & \\
		& & & {\LARGE \TikZ} & & \\
		& & & & {\LARGE \TeXLive} & \\
		& & & & & {\LARGE \hologo{MiKTeX}} \\
	\end{tblr}
}

\usepackage{hyperref}
\urlstyle{same}
\hypersetup{pdfborder=0 0 0}
\setlength{\parindent}{0pt}
\definecolor{LightGray}{gray}{0.9}

\usepackage{babel}
\AddThinSpaceBeforeFootnotes
\FrenchFootnotes

\usepackage{listings}

\usepackage{newverbs}
\newverbcommand{\motcletex}{\color{cyan!75!black}}{}
\newverbcommand{\packagetex}{\color{violet!75!black}}{}

\usepackage[most]{tcolorbox}
\tcbuselibrary{listingsutf8}
\newtcblisting{DemoCode}[1][]{%
	enhanced,width=0.95\linewidth,center,%
	bicolor,size=title,%
	colback=cyan!2!white,%
	colbacklower=cyan!1!white,%
	colframe=cyan!75!black,%
	listing options={%
		breaklines=true,%
		breakatwhitespace=true,%
		style=tcblatex,basicstyle=\small\ttfamily,%
		tabsize=4,%
		commentstyle={\itshape\color{gray}},
		keywordstyle={\color{blue}},%
		classoffset=0,%
		keywords={},%
		alsoletter={-},%
		keywordstyle={\color{blue}},%
		classoffset=1,%
		alsoletter={-},%
		morekeywords={center,justify,\lipsum},%
		keywordstyle={\color{violet}},%
		classoffset=2,%
		alsoletter={-},%
		morekeywords={\QuizzMillions,\QuizzPrendrePlace},%
		keywordstyle={\color{green!50!black}},%
		classoffset=3,%
		morekeywords={Largeur,Marge,Hauteurs,Couleur,CouleurFond,CouleurLettre,CouleurTexte,Bonne,Mauvaise,Choix,CoulBonne,CoulChoix,CoulMauvaise,Affichage,EspacementV,OffsetV,Decorations,AffChoix,Jokers,AffJokers,CodeAvant,Offset,Theme,Type,Effet},%
		keywordstyle={\color{orange}}
	},%
	#1
}

\tcbset{vignettes/.style={%
	nobeforeafter,box align=base,boxsep=0pt,enhanced,sharp corners=all,rounded corners=southeast,%
	boxrule=0.75pt,left=7pt,right=1pt,top=0pt,bottom=0.25pt,%
	}
}

\tcbset{vignetteMaJ/.style={%
	fontupper={\vphantom{pf}\footnotesize\ttfamily},
	vignettes,colframe=purple!50!black,coltitle=white,colback=purple!10,%
	overlay={\begin{tcbclipinterior}%
			\fill[fill=purple!75]($(interior.south west)$) rectangle node[rotate=90]{\tiny \sffamily{\textcolor{black}{\scalebox{0.66}[0.66]{\textbf{MàJ}}}}} ($(interior.north west)+(5pt,0pt)$);%
	\end{tcbclipinterior}}
	}
}

\newcommand\Cle[1]{{\small\sffamily\textlangle \textcolor{orange}{#1}\textrangle}}
\newcommand\cmaj[1]{\tcbox[vignetteMaJ]{#1}\xspace}

\begin{document}

\setlength{\aweboxleftmargin}{0.07\linewidth}
\setlength{\aweboxcontentwidth}{0.93\linewidth}
\setlength{\aweboxvskip}{8pt}

\pagestyle{fancy}

\thispagestyle{empty}

\vspace{2cm}

\begin{center}
	\begin{minipage}{0.75\linewidth}
	\begin{tcolorbox}[colframe=yellow,colback=yellow!15]
		\begin{center}
			\begin{tabular}{c}
				{\Huge \texttt{quizztex} [fr]}\\
				\\
				{\LARGE Des présentations de Quizz,} \\
				\\
				{\LARGE à la manière de Jeux TV.} \\
			\end{tabular}
			
			\bigskip
			
			{\small \texttt{Version \TPversion{} -- \TPdate}}
		\end{center}
	\end{tcolorbox}
\end{minipage}
\end{center}

\begin{center}
	\begin{tabular}{c}
	\texttt{Cédric Pierquet}\\
	{\ttfamily c pierquet -- at -- outlook . fr}\\
	\texttt{\url{https://github.com/cpierquet/quizztex}}
\end{tabular}
\end{center}

\vspace{0.25cm}

{$\blacktriangleright$~~Créer des quizz à la manière de Jeux télévisés}

\vspace{0.25cm}

{$\blacktriangleright$~~Style \og Qui veut gagner des millions ? \fg{} ou \og Tout le monde veut prendre sa place ! \fg{}.}

\vspace{1cm}

\hfill\QuizzMillions[Largeur=12,Bonne=C,Mauvaise=B,Choix=B,Affichage=PropSol,AffJokers=50.TEL]
{Quelle partie de l'œuf dois-je cuisiner pour préparer avec succès une mayonnaise ?}
{Le rouge} {Le violet}
{Le blue} {Le jaune}\hfill~

\vspace{1cm}

\hfill\QuizzPrendrePlace[Largeur=12,Choix=C,Affichage=Choix]
{Quelle partie de l'œuf dois-je cuisiner pour préparer avec succès une mayonnaise ?}
{Le violet} {Le bleu}
{Le jaune} {Le rouge}\hfill~



\vspace{0.5cm}

%\hfill{}\textit{Merci à Denis Bitouzé et à Gilles Le Bourhis pour leurs retours et idées !}

\smallskip

\vfill

\hrule

\medskip

\TableauDocumentation

\medskip

\hrule

\medskip

\newpage

\phantomsection
\hypertarget{matoc}{}

\tableofcontents

\vfill

\section{Historique}

\verb|v0.1.0|~:~~~~Version initiale.

\newpage

\section{Le package quizztex}

\subsection{Introduction}

\begin{noteblock}
Le package propose de quoi afficher, dans son document \LaTeX, un quizz \textit{à la manière} d'un Jeu TV (créé à l'aide de \packagetex!tikz!), avec la possibilité :

\begin{itemize}
	\item choisir un style (\textsf{Millions} ou \textsf{PrendreSaPlace}) ;
	\item de spécifier les dimensions, la couleur, de rajouter un petit effet visuel sur les couleurs ;
	\item de personnaliser les choix (réponse, mauvaise, bonne, type, \ldots).
\end{itemize}
\vspace*{-\baselineskip}\leavevmode
\end{noteblock}

\subsection{Chargement du package, packages utilisés}

\begin{importantblock}
Le package se charge, de manière classique, dans le préambule.

Il n'existe pas d'option pour le package, et \packagetex!xcolor! n'est pas chargé.
\end{importantblock}

\begin{DemoCode}[listing only]
\documentclass{article}
\usepackage{quizztex}

\end{DemoCode}

\begin{noteblock}
\packagetex!quizztex! charge les packages suivantes :

\begin{itemize}
	\item \packagetex!tikz!, \packagetex!pgf! et \packagetex!pgffor! ;
	\item \packagetex!xstring!, \packagetex!simplekv!, \packagetex!settobox!, \packagetex!varwidth! et \packagetex!fontawesome5! ;
	\item les librairies \packagetex!tikz! :
	\begin{itemize}
		\item \packagetex!tikz.calc!
		\item \packagetex!tikz.positioning! ;
		\item \packagetex!tikz.shapes.geometric! ;
		\item \packagetex!tikz.babel! ;
		\item \packagetex!tikz.fadings!.
	\end{itemize}
\end{itemize}

Il est compatible avec les compilations usuelles en \textsf{latex}, \textsf{pdflatex}, \textsf{lualatex} ou \textsf{xelatex}.
\end{noteblock}

\subsection{Gestion des couleurs}

\begin{tipblock}
Des couleurs prédéfinies (type \textsf{HTML}) sont créées par le package \packagetex!quizztex!, afin de pouvoir gérer -- en interne -- des tracés avec des couleurs du type \motcletex!<couleur>!...!.
\end{tipblock}

\vfill~

\pagebreak

\section{Quizz à la manière de \og Qui veut gagner des Millions ? \fg}

\subsection{Commande et fonctionnement global}

\begin{cautionblock}
L'environnement dédié à la création du Quizz \og Millions \fg{} \motcletex!\QuizzMillions!.

\smallskip

\textsf{Who Wants to Be a Millionaire ?\texttrademark} est une marquée déposée de Sony Pictures Television.
\end{cautionblock}

\begin{DemoCode}[listing only]
\QuizzMillions[clés]%
	{Question}
	{Réponse A}
	{Réponse B}
	{Réponse C}
	{Réponse D}
\end{DemoCode}

\begin{DemoCode}[text only]
\QuizzMillions
	{Question}
	{Réponse A}
	{Réponse B}
	{Réponse C}
	{Réponse D}
\end{DemoCode}

\begin{tipblock}
Les éventuelles couleurs choisies devront être données de manière \textit{unique}, sans utiliser les \textit{mélanges} (avec \motcletex|CouleurA!...!CouleurB|) que propose le package \packagetex!xcolor!.

Toutefois, toute couleur précédemment définie pourra être utilisée pour le Quizz (c'est ce que propose les couleurs par défaut de \packagetex!quizztex!).
\end{tipblock}

\begin{importantblock}
Le code se charge d'ajuster la hauteur des cartouches, et la hauteur des cartouches \textit{Réponses} auront tous la même hauteur.

\smallskip

Il est cependant possible de préciser une hauteur manuelle globale, ou une hauteur pour le cartouche \textit{Question} et une hauteur pour les cartouches \textit{Réponses}.
\end{importantblock}

\subsection{Couleurs prédéfinies}

\begin{tipblock}
Les couleurs (HTML) définies et utilisées par le package \packagetex!quizztex! et pour l'environnement \motcletex!\QuizzMillions! sont :

\begin{itemize}
	\item \verb!\definecolor{FondWWTBAM}{HTML}{5E57A4}    ! : \textcolor{FondWWTBAM}{\textsf{\textbf{Fond par défaut}}}
	\item \verb!\definecolor{BonneWWTBAM}{HTML}{0AC759}   ! : \textcolor{BonneWWTBAM}{\textsf{\textbf{Fond de la bonne réponse}}}
	\item \verb!\definecolor{MauvaiseWWTBAM}{HTML}{F1901C}! : \textcolor{MauvaiseWWTBAM}{\textsf{\textbf{Fond de la mauvaise réponse}}}
	\item \verb!\definecolor{CouleurWWBTAM}{HTML}{140676} ! : \textcolor{CouleurWWBTAM}{\textsf{\textbf{Bordure}}}
	\item \verb!\definecolor{ChoixWWBTAM}{HTML}{F40FDC}   ! : \textcolor{ChoixWWBTAM}{\textsf{\textbf{Fond de la réponse choisie}}}
\end{itemize}

L'utilisateur qui souhaite modifier les couleurs devra être attentif à la coordination de celles-ci, afin d'obtenir un affichage pertinent et cohérent.
\end{tipblock}

\begin{DemoCode}[]
\QuizzMillions{\lipsum[1][1-2]}
	{\lipsum[2][1]}{\lipsum[2][2]}{\lipsum[2][3]}{\lipsum[2][5]}
\end{DemoCode}

\subsection{Clés et options}

\begin{tipblock}
Le premier argument, optionnel et entre \texttt{[...]}, propose les \Cle{clés} suivantes :

\begin{itemize}
	\item \Cle{Largeur} := largeur (sans unité, mais en cm) sans les Jokers, du Quizz ; \hfill{}défaut : \Cle{14}
	\item \Cle{Marge} := marge gauche/droite (sans unité, mais en cm, des petits traits horizontaux) ;
	
	\hfill{}défaut : \Cle{0.5}
	\item \Cle{Hauteurs} := hauteurs des cartouches (\motcletex!auto! ou \motcletex!global! ou \motcletex!Quest/Réponses!) ;
	
	\hfill{}défaut : \Cle{auto}
	\item \Cle{Couleur} := couleur des bordures ; \hfill{}défaut : \Cle{CouleurWWBTAM}
	\item \Cle{CouleurFond} := couleur du fond ; \hfill{}défaut : \Cle{FondWWTBAM}
	\item \Cle{CouleurLettre} := couleur des lettres ; \hfill{}défaut : \Cle{LettreWWBTAM}
	\item \Cle{CouleurTexte} := couleur des textes ; \hfill{}défaut : \Cle{white}
	\item \Cle{Bonne} := bonne réponse (A/B/C/D) ; \hfill{}défaut : \Cle{}
	\item \Cle{Mauvaise} := mauvaise réponse (A/B/C/D) ; \hfill{}défaut : \Cle{}
	\item \Cle{Choix} := réponse choisie (A/B/C/D) ; \hfill{}défaut : \Cle{}
	\item \Cle{CoulBonne} := couleur de la bonne réponse ; \hfill{}défaut : \Cle{BonneWWTBAM}
	\item \Cle{CoulChoix} := couleur du choix ; \hfill{}défaut : \Cle{ChoixWWBTAM}
	\item \Cle{CoulMauvaise} := couleur de la mauvaise réponse ; \hfill{}défaut \Cle{MauvaiseWWTBAM}
	\item \Cle{Affichage} := type d'affichage, parmi \motcletex!Choix/Sol/PropSol! ; \hfill{}défaut \Cle{}
	\item \Cle{EspacementV} := espacement vertical entre les cartouches ; \hfill{}défaut \Cle{8pt}
	%\item \Cle{OffsetV} := =6pt,%
	\item \Cle{Decorations} := booléen pour rajouter les petits \textit{diamants} des réponses  ; \hfill{}défaut \Cle{true}
	\item \Cle{AffChoix} := réponses à afficher (pour le 50/50 par exemple)  ; \hfill{}défaut \Cle{ABCD}
	\item \Cle{Jokers} := booléen pour afficher les Jokers ; \hfill{}défaut \Cle{true}
	\item \Cle{AffJokers} : = Jokers à marquer comme disponibles ; \hfill{}défaut \Cle{50.TEL.PUB}
	\item \Cle{CodeAvant} : = code à appliqer à tous les cartouches ; \hfill{}défaut \Cle{\textbackslash bfseries\textbackslash large\textbackslash sffamily}
	\item \Cle{Effet} := booléen pour utiliser un petit effet de dégradé.\hfill{}défaut \Cle{true}
\end{itemize}
\vspace*{-\baselineskip}\leavevmode
\end{tipblock}

\begin{tipblock}
Les cinq arguments obligatoires correspondent à la question et aux réponses, sans oublier que la clé \motcletex!CodeAvant! sera appliquée pour chacun de ces cinq arguments.
\end{tipblock}

\subsection{Exemples}

\begin{DemoCode}[]
%par défaut
\QuizzMillions
	{Quelle partie de l'\oe{}uf dois-je cuisiner pour préparer avec succès une mayonnaise ?}
	{Le violet} {Le bleu}
	{Le jaune} {Le rouge}
\end{DemoCode}

\begin{DemoCode}[]
%hauteurs manuelles
\QuizzMillions[Hauteurs=2cm/1.5cm]
	{Quelle partie de l'\oe{}uf dois-je cuisiner pour préparer avec succès une mayonnaise ?}
	{Le violet} {Le bleu}
	{Le jaune} {Le rouge}
\end{DemoCode}

\begin{DemoCode}[]
%sans effet et sans Jokers, largeur réduite
\QuizzMillions[Effet=false,Jokers=false,Largeur=9]
	{Quelle partie de l'\oe{}uf dois-je cuisiner pour préparer avec succès une mayonnaise ?}
	{Le violet} {Le bleu}
	{Le jaune} {Le rouge}
\end{DemoCode}

\begin{DemoCode}[]
%avec réponse choisie
\QuizzMillions[Bonne=C,Mauvaise=B,Choix=B,Affichage=Choix]
	{Quelle partie de l'\oe{}uf dois-je cuisiner pour préparer avec succès une mayonnaise ?}
	{Le violet} {Le bleu}
	{Le jaune} {Le rouge}
\end{DemoCode}

\begin{DemoCode}[]
%avec bonne réponse
\QuizzMillions[Bonne=C,Mauvaise=B,Choix=B,Affichage=Sol]
	{Quelle partie de l'\oe{}uf dois-je cuisiner pour préparer avec succès une mayonnaise ?}
	{Le violet} {Le bleu}
	{Le jaune} {Le rouge}
\end{DemoCode}

\begin{DemoCode}[]
%avec réponse fausse choisie et bonne réponse
\QuizzMillions[Bonne=C,Mauvaise=B,Choix=B,Affichage=PropSol]
	{Quelle partie de l'\oe{}uf dois-je cuisiner pour préparer avec succès une mayonnaise ?}
	{Le violet} {Le bleu}
	{Le jaune} {Le rouge}
\end{DemoCode}

\begin{DemoCode}[]
%avec jokers déjà utlisés et 50:50
\QuizzMillions[AffJokers=PUB.TEL,AffChoix=AC]
	{Quelle partie de l'\oe{}uf dois-je cuisiner pour préparer avec succès une mayonnaise ?}
	{Le violet} {Le bleu}
	{Le jaune} {Le rouge}
\end{DemoCode}

\begin{DemoCode}[]
%avec couleurs modifiées et police modifiée
\QuizzMillions[Couleur=black,CouleurFond=gray,CodeAvant={\LARGE\ttfamily}]
	{On considère la fonction $\mathtt{f}$ définie sur $\mathbb{R}$ par $\mathtt{f(x)=2\,\text{e}^{2x}}$. On a :}
	{$\mathtt{f'(x)=4\,\text{e}^{2x}}$}
	{$\mathtt{f'(x)=2\,\text{e}^{2x}}$}
	{$\mathtt{f'(x)=2\,\text{e}^{2}}$}
	{$\mathtt{f'(x)=\dfrac{1}{x}}$}
\end{DemoCode}

\pagebreak

\section{Quizz à la manière de \og Tout le monde veut prendre sa place \fg.}

\subsection{Commande et fonctionnement global}

\begin{cautionblock}
L'environnement dédié à la création du Quizz \og PrendrePlace \fg{} \motcletex!\QuizzPrendrePlace!.

\smallskip

\textsf{Tout le monde veut prendre sa place\texttrademark} est une marquée déposée de Air Productions.
\end{cautionblock}

\begin{DemoCode}[listing only]
\QuizzPrendrePlace[clés]%
	{Question}
	{Réponse A}
	{Réponse B}
	{Réponse C}
	{Réponse D}
\end{DemoCode}

\begin{DemoCode}[text only]
\QuizzPrendrePlace
	{Question}
	{Réponse A}
	{Réponse B}
	{Réponse C}
	{Réponse D}
\end{DemoCode}

\begin{tipblock}
Les éventuelles couleurs choisies devront être données de manière \textit{unique}, sans utiliser les \textit{mélanges} (avec \motcletex|CouleurA!...!CouleurB|) que propose le package \packagetex!xcolor!.

Toutefois, toute couleur précédemment définie pourra être utilisée pour le Quizz (c'est ce que propose les couleurs par défaut de \packagetex!quizztex!).
\end{tipblock}

\begin{importantblock}
Le code se charge d'ajuster la hauteur des cartouches, et la hauteur des cartouches \textit{Réponses} auront tous la même hauteur.

\smallskip

Il est cependant possible de préciser une hauteur manuelle globale, ou une hauteur pour le cartouche \textit{Question} et une hauteur pour les cartouches \textit{Réponses}.
\end{importantblock}

\subsection{Couleurs prédéfinies}

\begin{tipblock}
Les couleurs (HTML) définies et utilisées par le package \packagetex!quizztex! et pour l'environnement \motcletex!\QuizzMillions! sont :

\begin{itemize}
	\item \verb!definecolor{FondTLMVPSP}{HTML}{4E52E3}     ! : \textcolor{FondTLMVPSP}{\textsf{\textbf{Fond par défaut}}}
	\item \verb!\definecolor{BonneTLMVPSP}{HTML}{00E519}   ! : \textcolor{BonneTLMVPSP}{\textsf{\textbf{Fond de la bonne réponse}}}
	\item \verb!\definecolor{MauvaiseTLMVPSP}{HTML}{FF9F3F}! : \textcolor{MauvaiseTLMVPSP}{\textsf{\textbf{Fond de la mauvaise réponse}}}
	\item \verb!\definecolor{CoulTLMVPSP}{HTML}{171A7A}    ! : \textcolor{CoulTLMVPSP}{\textsf{\textbf{Bordure}}}
	\item \verb!\definecolor{ChoixTLMVPSP}{HTML}{6DCFF6}   ! : \textcolor{ChoixTLMVPSP}{\textsf{\textbf{Fond de la réponse choisie}}}
\end{itemize}

L'utilisateur qui souhaite modifier les couleurs devra être attentif à la coordination de celles-ci, afin d'obtenir un affichage pertinent et cohérent.
\end{tipblock}

\begin{DemoCode}[]
\QuizzPrendrePlace{\lipsum[1][1-2]}
{\lipsum[2][1]}{\lipsum[2][2]}{\lipsum[2][3]}{\lipsum[2][5]}
\end{DemoCode}

\subsection{Clés et options}

\begin{tipblock}
Le premier argument, optionnel et entre \texttt{[...]}, propose les \Cle{clés} suivantes :

\begin{itemize}
	\item \Cle{Largeur} := largeur (sans unité, mais en cm) ; \hfill{}défaut : \Cle{14}
%	\item \Cle{MargeMilieu} := marge intérieure (sans unité) entre les cartouches \textsf{Réponses} ;
%	
%	\hfill{}défaut : \Cle{1}
	\item \Cle{Hauteurs} := hauteurs des cartouches (\motcletex!auto! ou \motcletex!global! ou \motcletex!Quest/Réponses!) ;
	
	\hfill{}défaut : \Cle{auto}
	\item \Cle{Couleur} := couleur des bordures ; \hfill{}défaut : \Cle{CouleurTLMVPSP}
	\item \Cle{CouleurFond} := couleur du fond ; \hfill{}défaut : \Cle{FondTLMVPSP}
	\item \Cle{CouleurLettre} := couleur des lettres ; \hfill{}défaut : \Cle{LettreTLMVPSP}
	\item \Cle{CouleurTexte} := couleur des textes ; \hfill{}défaut : \Cle{white}
	\item \Cle{Bonne} := bonne réponse (A/B/C/D) ; \hfill{}défaut : \Cle{}
	\item \Cle{Mauvaise} := mauvaise réponse (A/B/C/D) ; \hfill{}défaut : \Cle{}
	\item \Cle{Choix} := réponse choisie (A/B/C/D) ; \hfill{}défaut : \Cle{}
	\item \Cle{CoulBonne} := couleur de la bonne réponse ; \hfill{}défaut : \Cle{BonneTLMVPSP}
	\item \Cle{CoulChoix} := couleur du choix ; \hfill{}défaut : \Cle{ChoixTLMVPSP}
	\item \Cle{CoulMauvaise} := couleur de la mauvaise réponse ; \hfill{}défaut \Cle{MauvaiseWWTBAM}
	\item \Cle{Affichage} := type d'affichage, parmi \motcletex!Choix/Sol/PropSol! ; \hfill{}défaut \Cle{}
	\item \Cle{EspacementV} := espacement vertical entre les cartouches ; \hfill{}défaut \Cle{8pt}
	%\item \Cle{OffsetV} := =6pt,%
	\item \Cle{Theme} := pour afficher un cartouche \textsf{Thème} sous le Quizz ; \hfill{}défaut \Cle{}
	\item \Cle{Type} = type de réponse, parmi \motcletex!Duo/Carre/Cash! ; \hfill{}défaut \Cle{Carre}
	\item \Cle{CodeAvant} : = code à appliqer à tous les cartouches ; \hfill{}défaut \Cle{\textbackslash bfseries\textbackslash large\textbackslash sffamily}
	\item \Cle{Effet} := booléen pour utiliser un petit effet de dégradé.\hfill{}défaut \Cle{true}
\end{itemize}
\vspace*{-\baselineskip}\leavevmode
\end{tipblock}

\begin{tipblock}
Les cinq arguments obligatoires correspondent à la question et aux réponses, sans oublier que la clé \motcletex!CodeAvant! sera appliquée pour chacun de ces cinq arguments.
\end{tipblock}

\subsection{Exemples}

\begin{DemoCode}[]
%par défaut
\QuizzPrendrePlace
	{Quelle partie de l'\oe{}uf dois-je cuisiner pour préparer avec succès une mayonnaise ?}
	{Le violet} {Le bleu}
	{Le jaune} {Le rouge}
\end{DemoCode}

\begin{DemoCode}[]
%hauteurs manuelles, avec thème
\QuizzPrendrePlace[Hauteurs=2cm/1.5cm,Theme={Cuisine}]
	{Quelle partie de l'\oe{}uf dois-je cuisiner pour préparer avec succès une mayonnaise ?}
	{Le violet} {Le bleu}
	{Le jaune} {Le rouge}
\end{DemoCode}

\begin{DemoCode}[]
%sans effet et Duo (uniquement choix C/D), largeur réduite
\QuizzPrendrePlace[Effet=false,Type=Duo,Largeur=9,Theme={Cuisine}]
	{Quelle partie de l'\oe{}uf dois-je cuisiner pour préparer avec succès une mayonnaise ?}
	{} {}
	{Le jaune} {Le rouge}
\end{DemoCode}

\begin{DemoCode}[]
%avec réponse choisie
\QuizzPrendrePlace[Bonne=C,Mauvaise=B,Choix=B,Affichage=Choix,Theme={Cuisine}]
	{Quelle partie de l'\oe{}uf dois-je cuisiner pour préparer avec succès une mayonnaise ?}
	{Le violet} {Le bleu}
	{Le jaune} {Le rouge}
\end{DemoCode}

\begin{DemoCode}[]
%avec bonne réponse
\QuizzPrendrePlace[Bonne=C,Mauvaise=B,Choix=B,Affichage=Sol,Theme={Cuisine}]
	{Quelle partie de l'\oe{}uf dois-je cuisiner pour préparer avec succès une mayonnaise ?}
	{Le violet} {Le bleu}
	{Le jaune} {Le rouge}
\end{DemoCode}

\begin{DemoCode}[]
%avec réponse fausse choisie et bonne réponse
\QuizzPrendrePlace[Bonne=C,Mauvaise=B,Choix=B,Affichage=PropSol,Theme={Cuisine}]
	{Quelle partie de l'\oe{}uf dois-je cuisiner pour préparer avec succès une mayonnaise ?}
	{Le violet} {Le bleu}
	{Le jaune} {Le rouge}
\end{DemoCode}

\begin{DemoCode}[]
%avec couleurs modifiées et police modifiée
\QuizzPrendrePlace[Couleur=black,CouleurFond=gray,CodeAvant={\LARGE\ttfamily}, Theme={Cuisine}]
	{On considère la fonction $\mathtt{f}$ définie sur $\mathbb{R}$ par $\mathtt{f(x)=2\,\text{e}^{2x}}$. On a :}
	{$\mathtt{f'(x)=4\,\text{e}^{2x}}$}
	{$\mathtt{f'(x)=2\,\text{e}^{2x}}$}
	{$\mathtt{f'(x)=2\,\text{e}^{2}}$}
	{$\mathtt{f'(x)=\dfrac{1}{x}}$}
\end{DemoCode}


%
%\subsection{Fonctionnement des points d'ancrage}
%
%\begin{tipblock}
%En plus du Post-It, le package \packagetex!postit! crée huit points d'ancrage pour le Post-It, qui seront nommés :
%
%\begin{itemize}
%	\item \motcletex!(<nom>-N)!, \motcletex!(<nom>-E)!, \motcletex!(<nom>-S)! et \motcletex!(<nom>-O)! pour les points Nord/Est/Sud/Ouest ;
%	\item \motcletex!(<nom>-N-O)!, \motcletex!(<nom>-N-E)!, \motcletex!(<nom>-S-E)! et \motcletex!(<nom>-S-O)! pour les points Nord Est/Nord Ouest/\ldots.
%\end{itemize}
%\end{tipblock}
%
%\begin{DemoCode}[]
%\begin{center}
%\begin{PostIt}[Inclinaison=10,Attache=Non,Rendu=tikz,RappelPostIt=MaPetiteNote1]
%	\lipsum[1][1-2]
%\end{PostIt}
%\end{center}
%\end{DemoCode}
%
%\begin{tikzpicture}[remember picture,overlay]
%	\foreach \dir/\pos in {N-O/above left,N/above,N-E/above right,E/right, S-E/below right,S/below,S-O/below left,O/left} 
%	{%
%		\draw[draw=blue,fill=red] (MaPetiteNote1-\dir) circle[radius=2pt] node[text=gray,\pos,font=\scriptsize\ttfamily] {MaPetiteNote1-\dir};%
%	}
%\end{tikzpicture}
%
%\begin{DemoCode}[]
%\begin{PostIt}[RappelPostIt=NoteY]<center>
%	Ceci est un petit Post-It ! Pour rappeler par exemple que \[(a+b)^2=a^2+2ab+b^2.\]
%\end{PostIt}\\
%\begin{PostIt}[Rendu=tikz,Largeur=8cm,Couleur=blue,Inclinaison=-5,RappelPostIt=NoteZ]
%	\lipsum[1][1-2]
%\end{PostIt}
%
%\begin{tikzpicture}[remember picture,overlay]
%	\draw[very thick,->,>=latex] (NoteY-S)to[out=-90,in=90](NoteZ-N) ;
%\end{tikzpicture}
%\end{DemoCode}
%
%\subsection{Exemples}
%
%\begin{DemoCode}[]
%\begin{PostIt}%moteur de rendu tcbox (défaut)
%	[Couleur=cyan,Attache=Trombone,Largeur=10cm,Inclinaison=10]<center,right=1.5cm>
%\lipsum[1][1-3]
%\end{PostIt}
%\end{DemoCode}
%
%\begin{DemoCode}[]
%\hfill\begin{PostIt}%moteur de rendu tikz
%	[Rendu=tikz,Couleur=violet,Largeur=9cm,Inclinaison=-10,Attache=Trombone,
%	CouleurAttache=black,ExtraMargeDroite=1cm,Titre={Petit Titre},
%	PoliceTitre={\color{white}\bfseries\small\sffamily}]
%\lipsum[1][1-3]
%\end{PostIt}\hfill~
%\end{DemoCode}
%
%\begin{DemoCode}[]
%\hfill\begin{PostIt}%moteur de rendu tikzv2
%	[Rendu=tikzv2,Couleur=orange,Largeur=9cm,Inclinaison=-10,Attache=Scotch, 	Titre={Essai},
%	PoliceTitre={\color{blue!50!black}\bfseries\itshape\small\ttfamily}]
%\lipsum[1][1-3]
%\end{PostIt}\hfill~
%\end{DemoCode}
%
%\begin{DemoCode}[]
%%usepackage{wrapstuff}
%\begin{wrapstuff}[r,top=1]
%\begin{PostIt}[Inclinaison=5,Coin,Couleur=pink,CouleurAttache=blue,Bordure=false]
%\lipsum[1][1-2]
%\end{PostIt}
%\end{wrapstuff}
%
%\lipsum[1]
%\end{DemoCode}
%
%\begin{DemoCode}[]
%%usepackage{wrapstuff}
%\begin{wrapstuff}[r,top=1]
%\begin{PostIt}[Inclinaison=5,Rendu=tikz,Couleur=pink, CouleurAttache=blue,Bordure=false]
%\lipsum[1][1-2]
%\end{PostIt}
%\end{wrapstuff}
%
%\lipsum[1]
%\end{DemoCode}
%
%\begin{DemoCode}[]
%%usepackage{wrapstuff}
%\begin{wrapstuff}[r,top=1]
%\begin{PostIt}[Inclinaison=5,Rendu=tikzv2,Attache=Scotch,Couleur=pink]
%\lipsum[1][1-2]
%\end{PostIt}
%\end{wrapstuff}
%
%\lipsum[1]
%\end{DemoCode}
%
%\begin{DemoCode}[]
%Un petit Post-It aligné à droite, et centré verticalement :
%%
%\hfill\begin{PostIt}[Inclinaison=-10,Couleur=orange,Largeur=5cm,Hauteur=5cm, AlignementV=center,Coin,CouleurAttache=yellow, DecalAttache=-1,AlignementPostIt=center]
%
%\textsf{\small\lipsum[1][1-2]}
%\[\mathsf{\displaystyle\sum_{k=1}^{n} k = \dfrac{n(n+1)}{2}}\]
%\end{PostIt}
%\end{DemoCode}
%
%%\begin{DemoCode}[]
%%Un petit Post-It aligné à droite, et centré verticalement :
%%%
%%\hfill\begin{PostIt}[Inclinaison=-10,Couleur=orange,Largeur=5cm,Hauteur=5cm, AlignementV=center,Rendu=tikz,Attache=Non,AlignementPostIt=center]
%%
%%\textsf{\small\lipsum[1][1-2]}
%%\[\mathsf{\displaystyle\sum_{k=1}^{n} k = \dfrac{n(n+1)}{2}}\]
%%\end{PostIt}
%%\end{DemoCode}
%%
%%\vfill~
%
%\pagebreak
%
%\section{Post-It simple, en ligne}
%
%\subsection{Commande et fonctionnement global}
%
%\begin{cautionblock}
%La commande dédiée à la création du \textit{mini-}Post-It est \motcletex!MiniPostIt!.
%
%Elle fonctionne sous forme autonome, avec uniquement la couleur en \Cle{option}.
%
%\smallskip
%
%Cette fois-ci le \textit{mini-} Post-It est créé à l'aide d'une commande \motcletex!tcbox!.
%
%\smallskip
%
%Les dimensions ne sont pas modifiables, et un \motcletex!\vphantom! est inséré au début de la \motcletex!tcbox! afin d'harmoniser la hauteur.
%\end{cautionblock}
%
%\begin{DemoCode}[listing only]
%\MiniPostIt(*)[couleur]{contenu}
%\end{DemoCode}
%
%\subsection{Arguments}
%
%\begin{noteblock}
%La version étoilée active l'ombre du \textit{mini-}Post-It.
%
%La couleur (\Cle{yellow}), est gérée par l'argument optionnel entre \texttt{[...]}.
%\end{noteblock}
%
%\subsection{Exemples}
%
%\begin{DemoCode}[]
%On va travailler sur une équation diophantienne du type $ax+by=c$.
%
%On va utiliser le \MiniPostIt*[orange]{théorème de Bezout}, le \MiniPostIt{théorème de Gauss} sans oublier la \MiniPostIt*[cyan]{réciproque}.
%
%Le schéma de résolution est classique, et assez simple à appréhender !
%\end{DemoCode}
%
%\pagebreak
%
%\section{Résumé des styles}
%
%\subsection{Moteur de rendu tcbox}
%
%\begin{DemoCode}[text only]
%\hfill\begin{PostIt}
%\texttt{Ombre/Épingle/Bordure}
%\end{PostIt}
%\begin{PostIt}[Ombre=false]
%\texttt{Épingle/Bordure}
%\end{PostIt}\hfill~
%
%\medskip
%
%\hfill\begin{PostIt}[Bordure=false]
%\texttt{Ombre/Épingle}
%\end{PostIt}
%\begin{PostIt}[Bordure=false,Ombre=false]
%\texttt{Épingle}
%\end{PostIt}\hfill~
%
%\medskip
%
%\hfill\begin{PostIt}[Attache=Trombone]
%\texttt{Ombre/Trombone/Bordure}\\
%~
%\end{PostIt}
%\begin{PostIt}[Attache=Scotch]
%\texttt{Ombre/Scotch/Bordure}\\
%~
%\end{PostIt}\hfill~
%
%\medskip
%
%\hfill\begin{PostIt}[Attache=Non]
%\texttt{Ombre/Bordure}
%\end{PostIt}
%\begin{PostIt}[Coin,Attache=Non]
%\texttt{Ombre/Bordure/Coin}
%\end{PostIt}\hfill~
%
%\vspace{1cm}
%
%\hfill\begin{PostIt}[Titre={Lipsum[1][1-4]},PoliceTitre={\large\sffamily},Inclinaison=5,Couleur=pink,Hauteur=6cm,Attache=Scotch,AlignementV=center,Coin]
%\lipsum[1][1-4]
%\end{PostIt}\hfill~
%\end{DemoCode}
%
%\pagebreak
%
%\subsection{Moteur de rendu tikz}
%
%\begin{DemoCode}[text only]
%\hfill\begin{PostIt}[Rendu=tikz]
%\texttt{Ombre/Épingle/Bordure}
%\end{PostIt}
%\begin{PostIt}[Ombre=false,Rendu=tikz]
%\texttt{Épingle/Bordure}
%\end{PostIt}\hfill~
%
%\medskip
%
%\hfill\begin{PostIt}[Bordure=false,Rendu=tikz]
%\texttt{Ombre/Épingle}
%\end{PostIt}
%\begin{PostIt}[Bordure=false,Ombre=false,Rendu=tikz]
%\texttt{Épingle}
%\end{PostIt}\hfill~
%
%\medskip
%
%\hfill\begin{PostIt}[Attache=Trombone,Rendu=tikz]
%\texttt{Ombre/Trombone/Bordure}\\
%~
%\end{PostIt}
%\begin{PostIt}[Attache=Scotch,Rendu=tikz]
%\texttt{Ombre/Scotch/Bordure}\\
%~
%\end{PostIt}\hfill~
%
%\medskip
%
%\hfill\begin{PostIt}[Attache=Non,Rendu=tikz]
%\texttt{Ombre/Bordure}
%\end{PostIt}\hfill~
%
%\vspace{1cm}
%
%\hfill\begin{PostIt}[Rendu=tikz,Titre={Lipsum[1][1-4]},PoliceTitre={\large\sffamily},Inclinaison=5,Couleur=pink,Hauteur=6cm,Attache=Scotch,AlignementV=center,Coin]
%\lipsum[1][1-4]
%\end{PostIt}\hfill~
%\end{DemoCode}
%
%\subsection{Moteur de rendu tikzv2}
%
%\begin{DemoCode}[text only]
%\hfill\begin{PostIt}[Rendu=tikzv2]
%\texttt{Ombre/Épingle/Bordure}
%\end{PostIt}
%\begin{PostIt}[Ombre=false,Rendu=tikzv2]
%\texttt{Épingle/Bordure}
%\end{PostIt}\hfill~
%
%\medskip
%
%\hfill\begin{PostIt}[Bordure=false,Rendu=tikzv2]
%\texttt{Ombre/Épingle}
%\end{PostIt}
%\begin{PostIt}[Bordure=false,Ombre=false,Rendu=tikzv2]
%\texttt{Épingle}
%\end{PostIt}\hfill~
%
%\medskip
%
%\hfill\begin{PostIt}[Attache=Trombone,Rendu=tikzv2]
%\texttt{Ombre/Trombone/Bordure}\\
%~
%\end{PostIt}
%\begin{PostIt}[Attache=Scotch,Rendu=tikzv2]
%\texttt{Ombre/Scotch/Bordure}\\
%~
%\end{PostIt}\hfill~
%
%\medskip
%
%\hfill\begin{PostIt}[Attache=Non,Rendu=tikzv2]
%\texttt{Ombre/Bordure}
%\end{PostIt}\hfill~
%
%\vspace{1cm}
%
%\hfill\begin{PostIt}[Rendu=tikzv2,Titre={Lipsum[1][1-4]},PoliceTitre={\large\sffamily},Inclinaison=5,Couleur=pink,Hauteur=6cm,Attache=Scotch,AlignementV=center,Coin]
%\lipsum[1][1-4]
%\end{PostIt}\hfill~
%\end{DemoCode}



\end{document}